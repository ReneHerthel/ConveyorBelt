%        File: main.tex
%     Created: Sat Jan 14 12:59  2017 C
% Last Change: Mon Jan 16 10:48  2017 C

\documentclass[a4paper, oneside]{scrreprt}
\usepackage{scrlayer-scrpage}

\usepackage[utf8]{inputenc}
\usepackage[ngerman]{babel}

\usepackage{hyperref}

\usepackage{minutes}
\usepackage{vhistory}

\clearscrheadings
\cfoot{v\vhCurrentVersion}
\ofoot{\pagemark}

% Set header
\pagestyle{scrheadings}
% Put header on chapter pages
\renewcommand*{\chapterpagestyle}{scrheadings}

\begin{document}
    \title{Protokoll}
    \author{\vhListAllAuthors}
    \date{Version \vhCurrentVersion\ vom \vhCurrentDate}
    \maketitle

    \begin{versionhistory}
        \vhEntry{0.1}{12.04.2017}{SJE}{Erster Entwurf}
    \end{versionhistory}

    \begin{Minutes}{Protokoll}
        \moderation{Jonas Fuhrman}
        \minutetaker{Stephan Jänecke}
        \minutesdate{2017-04-12}
        \starttime{8:50}
        \endtime{9:42}
        \participant{Alle} % TODO: Insert names
        \maketitle

        Die Anwesenheit wurde festgestellt.

        \topic{UML-Ordnerstruktur}
        Sebastian um Einhaltung der beschlossenen Ordnerstruktur. 

        \topic{Distancetracker}

        \topic{Interface}
        RegisterToChannel() wird aus Interface in Konstruktur
        verschoben.

        \topic{Threads}
        Nichts neues.

        \topic{Interfaces}
        Komponentendiagram hat Vorrang. Interfaces in
        Hardwarenahenklassen beginnen mit \texttt{I}.

        \topic{MainController}
        MsgHandler empfängt PulseMessage. Weitere Klasse PuckSpace
        beinhaltet alle Pucks auf der aktuelle Anlage. Alle Pucks in
        erhalten alle eingetroffenen Nachrichten.
        Puck bildet jeden Puck auf Automaten ab. Nachrichten werden
        von jedem Puck ausgewertet. Hier erfolgt auch die
        Fehlererkennung. Error handling erfolgt in anderer Klasse, die
        nur dem MsgHandler bekannt ist. PuckSpace fragt die Stati aller
        Pucks ab.
        Abhängig von den Stati der Pucks erfolgt die Steuerung der
        Anlage über MachineControl.

        \topic{HWaddresses}
        Enums und Defines für LEDs und Buttons.

        \topic{Serielle Schnittstelle}
        Protokoll muss definiert werden.

        \topic{Interprozesskommunikation}
        Interprozesskommunikation erfolgt über PulseMessages.

        \topic{Aufgaben}
        \task{Sebastian}{Definierte Interfaces implementieren}
        \task{Mathis, Stephan}{Control for LightSystem}
        \task{Jonas}{DistanceTracker}
        \task{Dennis}{Serielle Schnittstelle: Dokumentation und
        Implementierungskonzept}
    \end{Minutes}
\end{document}
