\chapter{Requirements und Use Cases}

\section{Systemebene}

Die Anforderungen aus der Aufgabenstellung sind nicht vollständig. Die
Struktur der nachfolgenden Kapitel soll sie bei der Strukturierung der
Analyse unterstützen. Dokumentieren Sie die Ergebnisse der Analysen
entsprechend.

\subsection{Stakeholder}

Ermitteln sie die Stakeholder für das Projekt und listen sie diese hier
auf.

\subsection{Anforderungen}

In der Aufgabenstellung sind Anforderungen an das System gestellt.
Arbeiten sie diese hier auf und ergänzen sie diese entsprechend der
Absprachen mit dem Betreuer. Achten sie auf die entsprechende
Atribuierung. 
Berücksichtigen sie auch mögliche Fehlbedienungen und Fehlverhalten des
Systems.

\subsection{Systemkontext}

Use Cases werden aus einer bestimmten Sicht erstellt. Dokumentieren sie
diese mittels Kontextdiagramm oder Use Case Diagramm. Die Use Cases und
Test Cases müssen zu der hier verwendeten Nomenklatur konsistent sein.

\subsection{Use Cases}

Dokumentieren sie hier, welche Use Cases sie auf der Systemebene
implementieren müssen. Die Test Cases sollen später zu den Use Cases
konsistent sein.

\section{Systemanalyse}
Ihr technisches System hat aus Sicht der Software bestimmte
Eigenschaften. Was muss man für die Entwicklung der Software in
Struktur, Schnittstellen, Verhalten und an Besonderheiten wissen? Wählen
sie eine Kapitelstruktur, die am besten zur Dokumentation ihrer
Ergebnisse geeignet ist.

\section{Softwareebene}

Sie sollen Software für die Steuerung des technischen Systems erstellen.
Aus den Anforderungen auf der Systemebene und der Systemanalyse ergeben
sich Anforderungen für Ihre Software. Insbesondere wird sich die
Software der beiden Anlagenteile in einigen Punkten unterscheiden.
Dokumentieren sie hier die Anforderungen, die sich speziell für die
Software ergeben haben.

\subsection{Systemkontex}

Wie sieht der Kontext Ihrer Software aus? Wie erfolgt die Kommunikation
mit Nachbarsystemen? Liste der ein- und ausgehenden Signale/Nachrichten.

\subsection{Anforderungen}

Welche wesentlichen Anforderungen ergeben sich aus den
Systemanforderungen für ihre Software? Achten sie auf die entsprechende
Atribuierung. Berücksichtigen sie auch mögliche Fehlbedienungen und
Fehlverhalten des Systems.
