\chapter{Teamorganisation}

Überlegen sie, welche Regeln sie für die Zusammenarbeit aufstellen
wollen und welche Rollen sie im Team verteilen wollen. Dokumentieren sie
diese hier zusammen mit weiteren Anmerkungen der Teamorganisation.
Listen oder Tabellen sind zum Beispiel ein kompakte und übersichtliche
Darstellungsformen für diesen Bereich.

\section{Verantwortlichkeiten}

Bennen sie Verantwortliche innerhalb des Projekts (Projektleiter,
Tester, Implementierer, etc.). Auch hier ist eine Listen- oder
Tabellendarstellung angebracht.

\section{Absprachen}

Listen sie hier die Absprachen im Team auf, z. B. Jour Fixe,
Kommunikation, Respond-Latenz, \ldots

\section{Repository-Konzept}

Überlegen sie sich, wie sie das Repository und die Ordner organisieren
wollen. Welche Regeln wollen sie beim Umgang mit Branches,
Auslieferungen, Nachrichten an den Commits usw. im Team einhalten?
Listen sie diese Absprachen hier auf. Überlegen Sie auch, wie die
Arbeitsabläufe sein sollen bei der Umsetzung von Arbeitsaufträgen oder
bei der Behebung von Fehlern.
