\chapter{Design}

Anmerkung: Die Implementierung MUSS zu Ihrem Design-Modell konsistent
sein. Strukturen, Verhalten und Bezeichner im Code müssen mit dem Modell
übereinstimmen. Daher ist ein wohlüberlegtes Design wichtig.

\section{System Architektur}

Erstellung sie eine Architektur für Ihre Software. Geben sie eine kurze
Beschreibung Ihrer Architektur mit den dazugehörenden Komponenten und
Schnittstellen an. Dokumentieren sie hier wichtige technische
Entscheidungen. Welche Pattern werden gegebenenfalls verwendet? Wie
erfolgt die interne Kommunikation?

\section{Datenmodellierung}

Bestimmen sie das Datenmodell und dokumentieren sie es hier mit Hilfe
von UML Klassendiagrammen unter Beachtung der Designprinzipien. Die
Modelle können mit Hilfe eines UML-Tools erstellt werden. Hier ist dann
ein Übersichtsbild einzufügen.
Geben sie eine kurze textuelle Beschreibung des Datenmodells und deren
wichtigsten Klassen und Methoden an.

\section{Verhaltensmodellierung}

Ihre Software muss zur Bearbeitung der Aufgaben ein Verhalten aufweisen.
Überlegen sie sich dieses Verhalten auf Basis der Anforderungen und
modellieren sie das Verhalten unter Verwendung von Verhaltensdiagrammen.
Sie können für die Spezifikation der Prozess-Lenkung entweder
Petri-Netze oder hierarchische Automaten verwenden. Die Modelle können
mit Hilfe eines UML-Tools erstellt werden. Hier sind dann kommentierte
Übersichtsbilder einzufügen.
