\chapter{Testen}

\section{Test Konzept}
Das Testen soll nach dem V-Modell von der kleinsten zur Größten Einheit ablaufen.
Alle Klassen werden gegen Interfaces Implementiert (Design for Testability).

\subsection{Unit Test}
Nach dem fertig stellen einer Klasse sollte dieses mithilfe eines Unit Tests geprüft werden (simple Funktionen wie z.B setter und getter müssen nicht getestet werden).
Nach dem ändern dieser Klasse ist der Unit Test nochmals durchzuführen (Regressionstest).

\subsection{Komponenten Test}
Nach dem Fertigstellen einer Komponente soll diese, mithilfe des nach außen Sichtbaren Interfaces zuerst mit Stubs und Mocks (ohne Hardware) auf ihre Funktionalität geprüft werden.
Ist dieser Test erfolgreich, so wird der Test erneut auf der Hardware durchgeführt ohne Stub und Moch Objekte in der Komponente. 

Nach ändern der Komponente sind die Tests erneut durchzuführen.



\section{Testplan}

Definieren sie Zeitpunkte für die jeweiligen Teststufen in ihrer
Projektplanung. Dazu können sie die Meilensteine zu Hilfe nehmen.

\section{Abnahmetest}

Leiten sie die Abnahmebedingungen aus den Kunden-Anforderungen her.
Dokumentieren sie hier, welche Schritte für die Abnahme erforderlich
sind und welches Ergebnis jeweils erwartet wird (Test Cases).

\section{Testprotokolle und Auswertungen}

Hier fügen sie die Test Protokolle bei, auch wenn Fehler bereits
beseitigt worden sind, ist es schön zu wissen, welche Fehler einst
aufgetaucht sind. Eventuelle Anmerkung zur Fehlerbehandlung kann für
weitere Entwicklungen hilfreich sein.
Das letzte Testprotokoll ist das Abnahmeprotokoll, das bei der
abschließenden Vorführung erstellt wird. Es enthält eine Auflistung der
erfolgreich vorgeführten Funktionen des Systems sowie eine Mängelliste
mit Erklärungen der Ursachen der Fehlfunktionen und Vorschlägen zur
Abhilfe.
