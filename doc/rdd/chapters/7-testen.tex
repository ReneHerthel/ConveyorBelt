\chapter{Testen}

Machen sie sich auf Basis ihrer Überlegungen zur Qualitätssicherung
Gedanken darüber, wie sie die Erfüllung der Anforderungen möglichst
automatisiert im Rahmen von Unit-Test, Komponententest,
Integrationstest, Systemtest, Regressionstest und Abnahmetest überprüfen
werden.

\section{Testplan}

Definieren sie Zeitpunkte für die jeweiligen Teststufen in ihrer
Projektplanung. Dazu können sie die Meilensteine zu Hilfe nehmen.

\section{Abnahmetest}

Leiten sie die Abnahmebedingungen aus den Kunden-Anforderungen her.
Dokumentieren sie hier, welche Schritte für die Abnahme erforderlich
sind und welches Ergebnis jeweils erwartet wird (Test Cases).

\section{Testprotokolle und Auswertungen}

Hier fügen sie die Test Protokolle bei, auch wenn Fehler bereits
beseitigt worden sind, ist es schön zu wissen, welche Fehler einst
aufgetaucht sind. Eventuelle Anmerkung zur Fehlerbehandlung kann für
weitere Entwicklungen hilfreich sein.
Das letzte Testprotokoll ist das Abnahmeprotokoll, das bei der
abschließenden Vorführung erstellt wird. Es enthält eine Auflistung der
erfolgreich vorgeführten Funktionen des Systems sowie eine Mängelliste
mit Erklärungen der Ursachen der Fehlfunktionen und Vorschlägen zur
Abhilfe.
